\section{Déroulement du stage}
\subsection{Formation}

Les deux premières semaines de mon stage étaient dédiées à ma formation au progiciel Anadefi. Durant la première semaine, mon tuteur m'a présenté à son équipe, il m'a également présenté les locaux, comment est-ce qu'ils travaillaient tous ensemble et la présentation du progiciel entre autres.\\

La deuxième semaine, ma formation a eu lieu à Montpellier. J'ai tout d'abord installé un système de base de données (SQLServer) qui contenait toutes les données de leurs clients Anadefi. Par la suite, j'ai installé le progiciel et on m'a expliqué son mode de fonctionnement. La formation était très théorique car la pratique allait venir directement en travaillant sur les projets.

\subsection{Apprentissage par la pratique}

Dans mon cas, la meilleure façon d'apprendre était par la pratique car on applique directement les tâches, les concepts qui ont été expliquées juste avant. Et puis c'est aussi un très bon moyen d'apprendre quand les tâches sont plus ou moins répétitives comme cela a été le cas pour deux des quatre projets sur lesquels j'ai travaillé. Et puis pour être complètement formé au progiciel Anadefi il faut en moyenne trois mois, qui correspond quasiment à la durée de mon stage donc cette méthode était bien la meilleure solution.

\subsection{Planification} 

Avant même mon arrivée, mon tuteur avait déjà prévu les projets sur lesquels j'allais travailler, en fonction de la difficulté, du fait que je ne sois pas issu du monde de la finance, de mes points forts qui sont la programmation dont faire des scripts, que je puisse aussi travailler en autonomie et de la durée en jour pour que je puisse voir plusieurs aspects du rôle de fonctionnel.


\section{Environnement de travail}


\subsection{Présentation de l'équipe}

Le pôle service Anadefi est composé de :
\begin{itemize}
\item Alexandre LIGER : Responsable Pôle Service Anadefi
\item Irenee MANGENOT : Consultant / Chef de projet
\item Simon LABATUT : Presales Manager
\item Haseeb CHAUDHRY : Stagiaire consultant fonctionnel\\
\end{itemize}

Irenee MANGENOT est située à Montpellier dans les locaux de Sopra Banking Software (ex O.R System) tandis que Alexandre Liger et Simon LABATUT font régulièrement du télétravail mais se sont rendu très disponible pour me rejoindre sur Paris (La défense ou Kléber) au moins deux fois par semaine.
Me concernant, j'étais dans les locaux de La Défense (tour Manhattan) ou dans les locaux de Kléber en fonction de la venue d'Alexandre LIGER et/ou de Simon LABATUT.\\

Durant ce stage, j'ai donc travaillé la majeure partie de mon temps en autonomie et à distance bien que, je pouvais contacter mon tuteur ou l'un des responsable projets à tout moment quelle que soit ma demande.

\subsection{Méthodes de travail}

La méthodologie de travail utilisée est le cycle en V. La première étape de cette méthode permet de détailler le service jusqu'au début de sa réalisation. Il comprend donc les étapes suivantes: L'expression des besoins du client, l'analyse, écriture des spécifications générales et fonctionnelles. Une fois la réalisation du service achevé, on effectue une série de tests: test fonctionnel, d'intégration et enfin de validation qui va déterminer si le service répond au besoin exprimé par le client. Après cela, nous procédons à la livraison de la recette.\\

Sur tous les projets j'ai donc utilisé cette méthodologie et réalisé ou pris part a certaines de ces étapes.\\

Pour mon travail, un suivi quotidien était organisé sous forme de réunion via Skype/Teams avec mon tuteur. Je parlais de mon avancement effectué la veille, de mes éventuelles difficultés rencontrées. Je lui montrais mon travail et ce que j'allais faire durant la journée.\\

Certains aspects de la méthode agile ont pu être appliqués après ma demande auprès de mon tuteur comme le sprint planning meeting, le daily stand-up et la revue de sprint bien que la notion de sprint soit absente.

Malgré que je sois seul, il y avait toujours quelqu'un de disponible pour répondre à mes questions et valider le paramétrage ou l'écriture des spécifications.\\

Lorsqu'un paramétrage était terminé, j'envoyais le nouveau fichier de paramétrage à mon tuteur ou au responsable du projet en question pour qu'ils puissent à leur tour vérifier et le tester puis l'envoyer au client.\\